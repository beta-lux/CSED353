\documentclass{article}
\usepackage[T1]{fontenc}
\usepackage[utf8]{inputenc}
\usepackage{amsmath}
\usepackage{amsfonts}
\usepackage{fontspec}
\usepackage{amsthm}
\usepackage{amssymb}
\usepackage{stackengine}
\usepackage{minted}
\usepackage{mathtools}
\usepackage{graphicx,wrapfig,tikz}
\usepackage{physics}
\setlength\intextsep{0pt}
\usetikzlibrary{calc,hobby}
\usepackage{pgfplots}
\usepackage{xcolor}
\usepackage{xargs}
\usepackage{kotex}
\usepackage{mathrsfs} %fourier transform
\usetikzlibrary{external}
\usetikzlibrary{arrows}
\usepackage{setspace}
\usetikzlibrary{arrows}
\usepackage{tabstackengine}
\usepackage{enumitem}
\usepackage{longtable}
\usepackage{listings}
\setlist[enumerate]{itemsep=0mm}
\setlist[enumerate]{label={\arabic*.}}

\definecolor{beta-gray}{rgb}{0.9,0.9,0.9}

\newcommand{\lt}{\;\,}
\newcommand{\ft}{\quad\!}
\newcommand{\ot}{\quad\;}
\newcommand{\dt}{\;\,\;\,}
\newcommand{\plm}[1]{\setcounter{enumi}{#1 - 1}}
\newcommand{\nd}{\quad \text{and} \quad}
\newcommand{\inline}[1]{%
    \colorbox{beta-gray}{\lstinline[language=C++]{#1}}%
}

\newenvironment{code}[5]{
    {
        \usemintedstyle{xcode}
        \fontspec{Menlo}
        \begin{spacing}{1.0}
            \let\itshape\relax
            \inputminted[frame=single, firstline=#3, lastline=#4]{#1}{#2}
        \end{spacing}
        \vspace{#5}
    }

}

\newenvironment{beka}{

}

\begin{document}


    \author{\large 권민재, 20190084}
    \title{\Large\textbf{CSED353: Compouter Network Term Project}}
    \date{\small July 3, 2020}
    \maketitle
    \newpage

    \setstretch{1.5}
    \tableofcontents

    \newpage
    \section{개요}
     본 프로젝트는 single board computer와 여러 소프트웨어를 활용하여 unique WiFi AP를 제작하는 프로젝트이다.
     Raspberry Pi3 B 에 운영체제로 OpenWRT를 설치하고 여러 소프트웨어를 조합하여 WiFi AP를 제작하였다.

    \section{구조}

    \subsection{OpenWRT}
     OpenWRT는 wireless router를 위한 리눅스 기반의 오픈 소스 운영체제이다. 이 프로젝트에서 Raspberry Pi의 운영체제로 선택하여 사용했다.
     해당 운영체제는 '라우터'를 위한 운영체제라는 점에서 강점이 있다. 임베디드 보드에서 라우터 기능을 수행하기에는 성능적인 제약이 많기 때문에,
    꼭 필요한 기능만 설치되어 있고, 가벼운 운영체제가 적합하기 때문이다. 게다가, OpenWRT로 라우터를 만드는 사람들이 많이 존재하기 때문에,
     인터넷을 통해 많은 정보를 쉽게 접할 수 있다는 장점 또한 존재한다. 그래서 이번 프로젝트의 운영체제로 OpenWRT를 선택하였다.
    \subsection{OpenVPN}
     OpenVPN은 VPN을 제공하는 프로토콜과 프로그램을 일컫는다. OpenVPN은 OpenSSL을 이용하여 통신을 암호화하기 때문에 안전하고, 많은 플랫폼에서 사용할 수 있다는 장점이 있어서
     OpenVPN을 이용하여 VPN 서버 기능을 이 WiFi AP에 탑재하였다.
    \subsection{Squid}
     Squid는 프록시 서버 역할을 수행하며, 부가적으로 웹 캐시의 기능도 한다.
     하나의 프로그램으로 프록시와 웹 캐시 기능을 동시에 제공할 수 있다는 장점이 있어서 Squid를 이용하여 WiFi AP에 프록시 기능을 탑재하였다.
    \subsection{Nginx}
     Nginx는 최근 많이 사용되는 웹 서버 소프트웨어이다. Nginx는 다른 웹 서버 소프트웨어보다 강력한 리버스 프록시 기능을 제공한다.
     Nginx의 해당 기능을 통해 WiFi AP의 사용자들에게 *.postech.ac.kr에 대한 웹 캐시를 제공하고자 한다.
    \section{기능}
    모든 파일의 내용은 설정 파일들 중 중요하다고 생각되는 일부 부분들만 발췌한 것이다.
    \subsection{Wifi Routed AP}
     우선 lan과 wifi, 두 개의 인터페이스를 만들고, wifi에게 서브넷을 부여하였다. 그 후, wifi의 패킷을 lan으로 포워딩시키고,
     lan에서 MASQUERADE를 통해 nat를 해석하게 만들어서 Routed WiFi AP 기능을 구현하였다.
    \subsubsection{Interface}
    \textbf{lan}\\
    \code{cpp}{network}{10}{17}{5pt}
    lan 인터페이스는 Raspberry Pi에 유선으로 연결된 인터넷에 연결되는 인터페이스로서, 외부 네트워크와 연결되는 역할을 한다.
    학교 내에서 사용할 예정이기 때문에, 임시로 학교 내의 static ip를 할당하였다. \\ \\
    \textbf{wifi}\\
    \code{cpp}{network}{19}{22}{5pt}
    wifi 인터페이스는 Raspberry Pi가 무선으로 신호를 보낼 인터페이스를 말한다. 이 WiFi는 192.168.2.0/24의 내부 망 대역을 사용할 것이며,
    AP는 192.168.2.1을 할당받을 수 있도록 설정하였다.

    \subsubsection{dhcp}
    \code{cpp}{dhcp}{23}{32}{5pt}
    lan 인터페이스의 경우에는 static ip를 사용하고 있기 때문에, dhcp를 ignore 하도록 설정하였다.
    반면에, wifi의 경우, routed ap 기능을 수행하기 위해서는 dhcp에 기반한 nat가 필수적이기 때문에 위와 같이 dhcp 서버를 설정하였다.

    \subsubsection{firewall}
    \code{cpp}{firewall}{11}{30}{5pt}
    기본적으로, wifi의 패킷을 lan으로 포워딩하도록 설정하여 wifi를 통해 무선 인터넷을 사용할 수 있도록 구축하였다.
    그 후, lan에서는 nat가 적용되어 있는 wifi의 패킷들을 해석할 수 있어야 하게 때문에,
    option masq '1'을 통해 MASQUERADE를 활성화하여 nat을 처리할 수 있도록 만들었다.



    \subsubsection{wireless}
    \code{cpp}{wireless}{10}{16}{5pt}
    wireless에서는 SSID와 암호화 방식을 설정해주었으며, 보안성을 갖추기 위해서 psk2 방식을 선택하였다. 'CENSORED'는 실제로 사용하고 있는 암호가 아니다.



    \subsection{VPN}
     VPN 시장에서 널리 사용되고 있는 OpenVPN을 활용하여 WiFi AP가 VPN Server의 역할 또한 수행할 수 있도록 만들었다.
    \subsubsection{openvpn}
    \code{cpp}{openvpn}{}{}{5pt}
    openvpn에서 주로 쓰는 포트인 1194번 포트에서 tcp 방식으로 vpn이 작동할 수 있도록 설정하였다.
    각종 인증서와 키는 easyrsa를 통해 발급받았으며, 10.8.0.0/24 서브넷에서 vpn client들이 각자의 ip 주소를 할당받도록 설정하였다.
    이를 위해 dhcp-option을 통해 dhcp가 작동되도록 하였고, 클라이언트들의 모든 트래픽을 받기 위해 "redirect-gateway def1" 옵션을 추가하였다.
    테스트의 편의를 위해 같은 클라이언트 인증서로 여러 기기가 접속하는 것을 허용하기 위해 duplicate\_cn을 활성화시켰으며, lzo를 통해 압축된 패킷을 주고받도록 하게 만들었다.
    \subsubsection{firewall.user}
    \code{cpp}{firewall.user}{}{}{5pt}
    openvpn으로 통신되는 데이터들을 포워딩하기 위한 규칙을 설정하였다. 10.8.0.0/24 서브넷에서 위에서 정의한 tun0과 eth0(lan) 사이에 패킷이 오고 갈 수 있도록 설정하였고,
    이때 내부 망의 ip를 해석하기 위해 MASQUERADE도 활성화하였다.

    \subsection{Proxy with Caching}
     Squid를 이용하여 WiFi AP에 프록시 기능을 추가하였으며, 해당 프록시에서 Web caching을 할 수 있도록 설정하였다.
    \subsubsection{squid}
    \code{cpp}{squid.conf}{}{}{5pt}
    우선 acl을 통해 통신을 허용할 포트와 로컬 서브넷을 지정해주었다.
    이를 기반으로 특정 포트에서 프록시 접속을 차단하고, cache manager는 로컬 서브넷에서만 접속할 수 있도록 설정하여 기본적인 보안을 챙겼다.
    refresh\_pattern을 통해 caching을 진행할 데이터의 형태를 지정하였으며, youtube도 캐싱이 가능하도록 설정하였다.
    /tmp/squid/cache에 총 4096MB의 캐시 메모리를 할당하였으며, 위의 규칙과 같이 캐시가 저장되도록 설정하였다.


    \subsection{Caching}
     dnsmasq.conf를 통해 *.postech.ac.kr로 가는 패킷을 192.168.2.1, 즉 WiFi AP로 우회시킨 다음에 nginx에서 캐시가 있다면 캐시를 돌려주고,
     아닐 경우에 nginx에서 proxy path를 수행하게 하여 http에 대한 웹 캐시를 구현하였다.
    \subsubsection{nginx.conf}
    \code{cpp}{nginx.conf}{}{}{5pt}
    기본 resolver는 포스텍의 DNS로 설정하였으며, /nginx-cache/POSTECH/에 *.postech.ac.kr 페이지들의 캐시들이 저장되도록 설정하였다.
    WiFi AP의 80포트로 *.postech.ac.kr에 대한 요청이 들어왔을 경우, proxy\_cache\_valid 한 경우에 캐시를 클라이언트에게 돌려주고,
    아닌 경우에는 원래 경로로 proxy\_pass를 수행하여 클라이언트의 요청을 처리한다.
    원래대로라면 웹 규약에 맞게 X-Forwarded-For 헤더를 추가하는 것이 맞지만, HTTP 400 ERROR가 계속적으로 발생하는 원인이 되어 활성화시키지 않았다.
    \subsubsection{dnsmasq.conf}
    \code{cpp}{dnsmasq.conf}{}{}{5pt}
    *.postech.ac.kr에 대해 WiFi router를 가리키도록 dns를 수정하여 *.postech.ac.kr를 향하는 패킷들이 라우터의 nginx로 들어오도록 만들었다.


    \section{토론 및 개선}
    \begin{itemize}
    \item \begin{beka}
              Squid를 이용하여, 프록시를 구현할 때 웹 캐시도 같이 구현했지만,
              그 용량이 실 사용량에 비해 너무 작아서 효율이 떨어질 것으로 생각된다.
              캐시를 저장할 외장 저장장치를 이용하여 해결할 수 있을 것이다.
    \end{beka}
    \item \begin{beka}
              프록시를 이용하여 WiFi AP에 접속한 후 *.postech.ac.kr 사이트를 이용한다면,
              Squid와 Nginx에서 같은 내용의 캐시를 동시에 쌓기 때문에 퍼포먼스에 손해가 있을 것으로 생각된다.
              nginx로 글로벌한 캐시를 마련하고, 프록시 환경에서의 웹 캐시를 삭제하는 것이 오히려 효율적일 것으로 예측된다.
    \end{beka}
    \item \begin{beka}
              SSL 인증서와 관련하여 문제가 있어서 https 연결에 대해 nginx 캐시를 구현하지 못한 것이 아쉽다.
    \end{beka}
    \end{itemize}
    \section{참고문헌}
    \begin{itemize}
        \item OpenWrt Documentation (https://openwrt.org/docs/start)
    \end{itemize}
\end{document}